\documentclass[a4paper,12pt,numbers=noenddot]{scrreprt} 
\usepackage{iftex}
\iftutex
\usepackage{fontspec}
\defaultfontfeatures{Ligatures={TeX}}
\setmainfont{Noto Serif}
\setsansfont{Noto Sans}
\setmonofont{Noto Sans Mono}
\else
\usepackage[T2A]{fontenc}
\fi
\usepackage{hyphenat}
\hyphenation{ма-те-ма-ти-ка вос-ста-нав-ли-вать}
\usepackage[english, russian]{babel}
\usepackage{lipsum}

\usepackage{graphicx}

\usepackage[german]{datetime2}

\usepackage[left=2cm, right=2cm, bottom=3cm, top=3cm]{geometry}

\usepackage[onehalfspacing]{setspace}

\usepackage[utf8]{inputenc}

\usepackage{float}

\usepackage{enumerate}

\usepackage{url}

\usepackage{hyperref}
\usepackage[nohyperlinks]{acronym}

\usepackage{graphicx}

\usepackage[]{xcolor}

\usepackage{footnote}

\usepackage{tikz}

\usepackage{pgfplots}

\usepackage{todonotes}

\usepackage{lipsum}

\usepackage[german]{babel}

\usepackage[backend=biber]{biblatex}
\addbibresource{Bibliography.bib}
\usepackage{amsmath}

\usepackage{wrapfig}

\usepackage{textgreek}

\usepackage{csquotes}

\usepackage{multirow}

\pgfplotsset{compat=1.18}

\usepackage{scrhack} 

\setlength{\marginparwidth}{2cm}

\usepackage{scrlayer-scrpage} 
\renewcommand*{\chapterpagestyle}{scrheadings} 
\clearpairofpagestyles

\ohead{\normalfont \today} 
\chead{\normalfont}
\ihead{\normalfont \rightmark}

\ofoot{\normalfont Page~\pagemark}
\cfoot{\normalfont}
\ifoot{\normalfont \myauthor} 

\KOMAoptions{headsepline=true,footsepline=true} 

\renewcommand*{\chapterheadstartvskip}{\vspace*{-.4cm}}
\renewcommand*{\chapterheadendvskip}{\vspace{.5cm}}

\setlength{\parindent}{0pt} 

\setkomafont{chapter}{\LARGE}
\setkomafont{section}{\Large}
\setkomafont{subsection}{\large}
\setkomafont{subsubsection}{\normalsize}
\setkomafont{paragraph}{\normalsize}
\setkomafont{subparagraph}{\small}
\newcommand{\myauthor}{Victoria Brovina}
\usepackage{forest}

\begin{document}

\automark{section}
\automark{chapter}

\chapter{Матпрак №7} 

\section{№5 Найти решение рекуррентного соотношения}

\begin{align}
    T_n &= 3T_{n-1} - 15, \quad T_1 = 15 \\
    T_1 &= 3T_0 - 15 \Rightarrow T_0 = 10
\end{align}

\begin{flushleft}
Воспользуемся методом производящих функций. Данное соотношение имеет порядок = 1, так как для вычисления $T_n$ необходимо знать $T_{n-1}$.
Нам необходимо найти производящую функцию последовательности вида
\begin{align} 
    G(z) = \sum_{n=0}^{\infty} T_n z^n = T_0 + T_1 z + T_2 z^2 + \cdots
\end{align}
Для этого умножим $T_0$ на $z^0$ (далее умножали бы $T_1$ на $z^1$, ..., $T_{m-1}$ на $z^{m-1}$, где $m$ - порядок соотношения). Таким образом имеем:
\begin{align}
    & 1\cdot T_0=1\cdot 10 \\&
    z^n\cdot T_n=(3T_{n-1}-15)\cdot z^n, n \geq 1
\end{align}
Теперь сложим уравнения для всех значений $n$:
\begin{align}
    & T_0 + \sum_{n=1}^{\infty} z^n \cdot T_n = 10 + \sum_{n=1}^{\infty} z^n \cdot (3T_{n-1}-15) \\&
\end{align}
Левая часть = $G(z)$, а правая часть содержит суммы, похожие на $G(z)$. Приведем их к виду $G(z)$:
\begin{align}
    & \sum_{n=1}^{\infty} 3 \cdot z^n \cdot T_{n-1} = 3 \cdot z \cdot \sum_{n=1}^{\infty} z^{n-1} \cdot T_{n-1} = 3 \cdot z \cdot \sum_{n=0}^{\infty} z^{n} \cdot T_{n} = 3 \cdot z \cdot G(z) \\&
    \sum_{n=1}^{\infty} 15 \cdot z^n = 15 \cdot \sum_{n=1}^{\infty} z^n = 15 \cdot (\sum_{n=0}^{\infty} z^n - 1) = 15 \cdot (\frac{1}{1 - z} - 1) = \frac{15z}{1 - z}
\end{align}
Уравнение производящей функции примет следующий вид:
\begin{align}
    & G(z) = 10 + 3 \cdot z \cdot G(z) - \frac{15z}{1-z} \\&
    (1-3z) \cdot G(z) = 10 - \frac{15z}{1-z} \\&
    G(z)= \frac{10-25z}{(1-z)(1-3z)}
\end{align}
Разложим дробь на сумму простых дробей:
\begin{align}
    \frac{10-25z}{(1-z)(1-3z)} = \frac{15}{2(1-z)} + \frac{5}{2(1-3z)}
\end{align}
\end{flushleft}

\begin{flushleft}
Разложение рациональных функций:
\end{flushleft}

$\begin{aligned}
& \frac{1}{1-z} = \sum_{n=0}^{\infty} z^n = 1 + z + z^2 + z^3 + \cdots \\&
\frac{1}{1-3z} = \sum_{n=0}^{\infty} (3z)^n = 1 + 3z + 9z^2 + \cdots \\&
\end{aligned}$

\begin{flushleft}
Таким образом имеем:
\end{flushleft}

\begin{align}
G(z) = \frac{15}{2} \cdot \sum_{n=0}^{\infty} z^n + \frac{5}{2} \cdot \sum_{n=0}^{\infty} (3z)^n
\end{align}

\begin{flushleft}
Так как мы искали ответ вида $G(z)=\sum_{n=0}^{\infty} T_n \cdot z^n$, то решение рекуррентного соотношения (в силу равенства рядов):
\end{flushleft}

\begin{align}
T_n = \frac{15}{2} + \frac{5}{2} \cdot 3^n
\end{align}

\begin{flushleft}
Ответ: $T_n = \frac{15}{2} + \frac{5}{2} \cdot 3^n$.
\end{flushleft}

\section{№6 Найти решение рекуррентного соотношения}

\begin{align}
    T_n &= T_{n-1} + n - 1, \quad T_1 = 3 \\
    T_1 &= T_0 + 1 - 1 \Rightarrow T_0 = 3
\end{align}

\begin{flushleft}
Перейдем к решению:
\end{flushleft}

\begin{align}
& \sum_{n=2}^{\infty} z^n \cdot T_{n-1} = z \cdot \sum_{n=2}^{\infty} z^{n-1} \cdot T_{n-1} = z \cdot (\sum_{n=1}^{\infty} z^{n} \cdot T_n + T_0 - T_0) = z \cdot (G(z) - 3) \\&
\sum_{n=2}^{\infty} z^n \cdot (n - 1) = z \cdot \sum_{n=2}^{\infty} (n - 1) \cdot z^{n-1} = z \cdot (\sum_{n=0}^{\infty} n \cdot z^{n} - 1) = \frac{z^2}{(1-z)^2} \\&
G(z)=3+z\cdot G(z) + \frac{z}{1-z} \\&
G(z)=3+3z+\frac{z^2}{(z-1)^2}+z(G(z)-3) \\&
(1-z)G(z)=3+\frac{z^2}{(z-1)^2} \\&
G(z)=\frac{4z^2-6z+3}{(1-z)^3} \\&
\frac{4z^2-6z+3}{(1-z)^3}=\frac{4}{(1-z)}+\frac{-2}{(1-z)^2}+\frac{1}{(1-z)^3} \\&
\frac{1}{(1-z)}=\sum_{n=0}^{\infty} z^n \\&
\frac{1}{(1-z)^2}=\sum_{n=0}^{\infty} z^n \cdot (n+1) \\&
\frac{1}{(1-z)^3}=\sum_{n=0}^{\infty} z^n \cdot \frac{1}{2}(n+1)(n+2) \\&
T_n=4-2(n+1)+\frac{1}{2}(n+1)(n+2)=\frac{n^2}{2}-\frac{n}{2}+3
\end{align}

\begin{flushleft}
Ответ: $T_n = \frac{n^2}{2}-\frac{n}{2}+3$.
\end{flushleft}

\section{№7 Вычислить сумму}

\begin{align}
    A(k)=\frac{1}{2^k-1} \sum_{j=1}^{k} j \cdot 2^{j-1}
\end{align}

\begin{flushleft}
Перейдем к решению:
\end{flushleft}

\begin{align}
& \sum_{j=1}^{k} j \cdot 2^{j-1} = \frac{1}{2} \cdot \sum_{j=1}^{k} j \cdot 2^j \\&
S_k = \sum_{j=1}^{k} j \cdot 2^j \\&
S_k=2S_k-S_k=2 \cdot 2 + 2 \cdot 2 \cdot 2^2 + \cdots + 2 \cdot k \cdot 2^k - (2+2 \cdot 2 \cdot 2^2 + \cdots + k \cdot 2^k) = \\&
= -2 -2^2 - \cdots - 2^k + k \cdot 2^{k+1} = 2 - 2^{k+1} + k \cdot 2^{k+1} = (k-1) \cdot 2^{k+1} + 2 \\&
\frac{1}{2} \cdot S_k = 2^k \cdot (k-1) + 1 \\&
A(k) = \frac{1}{2^k-1} \cdot (2^k(k-1) + 1) = \frac{2^k \cdot (k-1) + 1}{2^k-1}
\end{align}

\begin{flushleft}
Ответ: $A(k) = \frac{2^k \cdot (k-1) + 1}{2^k-1}$.
\end{flushleft}

\section{№8  Найдите решение рекуррентного соотношения}

\begin{align}
    J_n = 6J_{n-1}+16J_{n-2}, \quad J_0=1, \quad J_1 = 7
\end{align}

\begin{flushleft}
Перейдем к решению:
\end{flushleft}

\begin{align}
& 1 \cdot J_0 = 1 \cdot 1 \\&
z \cdot J_1 = z \cdot 7 \\&
z^n \cdot J_n = (6J_{n-1}+16J_{n-2}) \cdot z^n, n \geq 2\\&
J_0 + z \cdot J_1 + z^n \cdot J_n = 1 + 7 \cdot z + z^n \cdot 6 \cdot J_{n-1} + z^n \cdot 16 \cdot J_{n-2} \\&
G(z) = 1 + 7z +6 \cdot \sum_{n=2}^{\infty} z^n \cdot J_{n-1} + 16  \cdot \sum_{n=2}^{\infty} z^n \cdot J_{n-2} \\&
\sum_{n=2}^{\infty} z^n \cdot J_{n-1} = z \cdot \sum_{n=2}^{\infty} z^{n-1} \cdot J_{n-1} = z \cdot \sum_{n=1}^{\infty} z^n \cdot J_n = z \cdot (\sum_{n=1}^{\infty} z^n \cdot J_n + J_0 - J_0) = \\&
 = z \cdot (G(z) - J_0) = z \cdot (G(z) - 1) \\&
\sum_{n=2}^{\infty} z^n \cdot J_{n-2} = z^2 \cdot \sum_{n=2}^{\infty} z^{n-2} \cdot J_{n-2} = z^2 \cdot \sum_{n=0}^{\infty} z^{n} \cdot J_{n} = z^2 \cdot G(z) \\&
G(z)= 1 + 7z + 6z(G(z)-1)+16z^2G(z) \\&
(1-16z^2-6z)G(z)=1+z \Rightarrow G(z) = \frac{1+z}{-16z^2-6z+1} \\&
G(z) = \frac{9}{10(1-8z)} + \frac{1}{10(1+2z)} \\&
G(z)= \frac{9}{10} \cdot \sum_{n=0}^{\infty} (8z)^n + \frac{1}{10} \cdot \sum_{n=0}^{\infty} (-2z)^n \\&
J_n=\frac{9}{10} \cdot 8^n + \frac{1}{10} \cdot (-2)^n
\end{align}

\begin{flushleft}
Ответ: $J_n=\frac{9}{10} \cdot 8^n + \frac{1}{10} \cdot (-2)^n$.
\end{flushleft}

\section{№9  Найти решение рекуррентного соотношения}

\begin{align}
    J_n = 3J_{n-1}+4J_{n-2}, \quad J_0=1, \quad J_1 = 3
\end{align}

\begin{flushleft}
Перейдем к решению:
\end{flushleft}

\begin{align}
& 1 \cdot J_0 = 1 \cdot 1 \\&
z \cdot J_1 = z \cdot 3 \\&
z^n \cdot J_n = (3J_{n-1}+4J_{n-2}) \cdot z^n, n \geq 2\\&
J_0 + z \cdot J_1 + z^n \cdot J_n = 1 + 3 \cdot z + z^n \cdot 3 \cdot J_{n-1} + z^n \cdot 4 \cdot J_{n-2} \\&
G(z) = 1 + 3z + 3 \cdot \sum_{n=2}^{\infty} z^n \cdot J_{n-1} + 4  \cdot \sum_{n=2}^{\infty} z^n \cdot J_{n-2} \\&
\sum_{n=2}^{\infty} z^n \cdot J_{n-1} = z \cdot \sum_{n=2}^{\infty} z^{n-1} \cdot J_{n-1} = z \cdot \sum_{n=1}^{\infty} z^n \cdot J_n = z \cdot (\sum_{n=1}^{\infty} z^n \cdot J_n + J_0 - J_0) = \\&
 = z \cdot (G(z) - J_0) = z \cdot (G(z) - 1) \\&
\sum_{n=2}^{\infty} z^n \cdot J_{n-2} = z^2 \cdot \sum_{n=2}^{\infty} z^{n-2} \cdot J_{n-2} = z^2 \cdot \sum_{n=0}^{\infty} z^{n} \cdot J_{n} = z^2 \cdot G(z) \\&
G(z)= 1 + 3z + 3z(G(z)-1)+4z^2G(z) \\&
(1-4z^2-3z)G(z)=1 \Rightarrow G(z) = \frac{1}{-4z^2-3z+1} \\&
G(z) = \frac{-4}{5(4z-1)} + \frac{1}{5(z+1)} \\&
G(z)= \frac{4}{5} \cdot \sum_{n=0}^{\infty} (-4z)^n + \frac{1}{5} \cdot \sum_{n=0}^{\infty} (-1z)^n \\&
J_n=\frac{4}{5} \cdot 4^n + \frac{1}{5} \cdot (-1)^n
\end{align}

\begin{flushleft}
Ответ: $J_n=\frac{4}{5} \cdot 4^n + \frac{1}{5} \cdot (-1)^n$.
\end{flushleft}

\section{№12 Найти асимптотическое решение рекуррентного соотношения}

\begin{flushleft}
    Для того чтобы найти асимптотическое решение рекуррентного соотношения воспользуемся методом построения дерева рекурсий. Для этого нарисуем дерево, которое на 0-м уровне совершит 1 вызов, на 1-м уровне 2, на 2-м уровне 4 и так далее:
\end{flushleft}

\begin{forest}
  for tree={circle, draw, minimum size=1.5em, edge={-latex}}
  [n
    [$\frac{n}{2}$
      [$\frac{n}{4}$
        [$\frac{n}{8}$
            [$...$]
            [$...$]
        ]
        [$\frac{n}{16}$
            [$...$]
            [$...$]
        ]
      ]
      [$\frac{n}{8}$
        [$\frac{n}{16}$
            [$...$]
            [$...$]
        ]
        [$\frac{n}{32}$
            [$...$]
            [$...$]
        ]
      ]
    ]
    [$\frac{n}{4}$
      [$\frac{n}{8}$
        [$\frac{n}{16}$
            [$...$]
            [$...$]
        ]
        [$\frac{n}{32}$
            [$...$]
            [$...$]
        ]
      ]
      [$\frac{n}{16}$
        [$\frac{n}{32}$
            [$...$]
            [$...$]
        ]
        [$\frac{n}{64}$
            [$...$]
            [$...$]
        ]
      ]
    ]
  ]
\end{forest}

\begin{flushleft}
    Как мы знаем рекурсия завершается по достижении базового кейса. В данном случае базовым кейсом будет $1=\frac{n}{2^i}$. Прологарифмируя получим $i=log_2(n)$, это высота построенного дерева.
\end{flushleft}

\begin{flushleft}
    Найдем суммы узлов на каждом из уровней:
\end{flushleft}

\begin{align}
& i=0: n \\&
i=1: \frac{n}{2}+\frac{n}{4}=\frac{3n}{4} \\&
i=2: \frac{n}{4}+\frac{n}{8}+\frac{n}{8}+\frac{n}{16}=\frac{9n}{16} \\&
i=3: \frac{n}{8}+\frac{n}{16}+\frac{n}{16}+\frac{n}{32} + \frac{n}{16}+\frac{n}{32}+\frac{n}{32}+\frac{n}{64}= \frac{27n}{64} \\&
\cdots \\&
i: (\frac{3}{4})^i\cdot n
\end{align}

\begin{flushleft}
Заметим, что на каждом из уровней сумма узлов всегда будет равна $n$. 
Тогда сложность рекурсивного вызова от $n$ будет $a_n=\sum_{n=0}^{i} n \cdot (\frac{3}{4})^i = n \cdot \sum_{n=0}^{log_2(n)} (\frac{3}{4})^i =n \cdot \frac{1}{1-\frac{3}{4}} = 4n$. Это и есть искомое асимптотическое решение.
\end{flushleft}

\begin{flushleft}
Ответ: $a_n=4n$.
\end{flushleft}

\section{№13 Найти асимптотическое решение рекуррентного соотношения}

\begin{flushleft}
    Для того чтобы найти асимптотическое решение рекуррентного соотношения воспользуемся методом построения дерева рекурсий. Для этого нарисуем дерево, которое на 0-м уровне совершит 1 вызов, на 1-м уровне 2, на 2-м уровне 4 и так далее:
\end{flushleft}

\begin{forest}
  for tree={circle, draw, minimum size=1.5em, edge={-latex}}
  [n
    [$\frac{n}{4}$
      [$\frac{n}{16}$
        [$\frac{n}{64}$
            [$...$]
            [$...$]
        ]
        [$\frac{3n}{64}$
            [$...$]
            [$...$]
        ]
      ]
      [$\frac{3n}{16}$
        [$\frac{3n}{64}$
            [$...$]
            [$...$]
        ]
        [$\frac{9n}{64}$
            [$...$]
            [$...$]
        ]
      ]
    ]
    [$\frac{3n}{4}$
      [$\frac{3n}{16}$
        [$\frac{3n}{64}$
            [$...$]
            [$...$]
        ]
        [$\frac{9n}{64}$
            [$...$]
            [$...$]
        ]
      ]
      [$\frac{9n}{16}$
        [$\frac{9n}{64}$
            [$...$]
            [$...$]
        ]
        [$\frac{27n}{64}$
            [$...$]
            [$...$]
        ]
      ]
    ]
  ]
\end{forest}

\begin{flushleft}
    Как мы знаем рекурсия завершается по достижении базового кейса. В данном случае базовым кейсом будет $1=\frac{n}{4^i}$. Прологарифмируя получим $i=log_4(n)$, это высота построенного дерева.
\end{flushleft}

\begin{flushleft}
    Найдем суммы узлов на каждом из уровней:
\end{flushleft}

\begin{align}
& i=0: n \\&
i=1: \frac{n}{4}+\frac{3n}{4}=n \\&
i=2: \frac{n}{16}+\frac{3n}{16}+\frac{3n}{16}+\frac{9n}{16}=\frac{n}{4}+\frac{12n}{16}=n \\&
\cdots \\&
i: n
\end{align}

\begin{flushleft}
Заметим, что на каждом из уровней сумма узлов всегда будет равна $n$. 
Тогда сложность рекурсивного вызова от $n$ будет $a_n=\sum_{n=0}^{i} n = n \cdot \sum_{n=0}^{log_4(n)} 1 = n \cdot (log_4(n)+1)$. Это и есть искомое асимптотическое решение.
\end{flushleft}

\begin{flushleft}
Ответ: $a_n=n \cdot (log_4(n)+1)$.
\end{flushleft}

\section{№17 Найти производящую функцию последовательности и решение рекуррентного соотношения}

\begin{align}
    a_n=6a_{n-1}-12a_{n-2}+18a_{n-3}-27a_{n-4}, \quad a_0=0, \quad a_1=a_2=a_3 = 1
\end{align}

\begin{flushleft}
22222Перейдем к решению:

\begin{align}
    & 1 \cdot a_0 = 1 \cdot 0 \\&
    z \cdot a_1 = z \cdot 1 \\&
    z^2 \cdot a_2 = z^2 \cdot 1 \\&
    z^3 \cdot a_3 = z^3 \cdot 1 \\&
    z^n \cdot a_n = (6a_{n-1}-12a_{n-2}+18a_{n-3}-27a_{n-4}) \cdot z^n, n \geq 4\\&
    a_1 z + a_2 z^2 + a_3 z^3 + z^n \cdot a_n = z + z^2 + z^3 + z^n \cdot (6a_{n-1}-12a_{n-2}+18a_{n-3}-27a_{n-4}) \\&
    a_1 z + a_2 z^2 + a_3 z^3 + \sum_{n=4}^{\infty} z^n \cdot a_n = z + z^2 + z^3 + \sum_{n=4}^{\infty} z^n \cdot (6a_{n-1}-12a_{n-2}+18a_{n-3}-27a_{n-4}) \\&
    G(z) = z + z^2 + z^3 + \sum_{n=4}^{\infty} z^n (6a_{n-1}-12a_{n-2}+18a_{n-3}-27a_{n-4}) \\&
    \sum_{n=4}^{\infty} z^n \cdot a_{n-1} = z \cdot \sum_{n=4}^{\infty} z^{n-1} \cdot a_{n-1} = z \cdot \sum_{n=3}^{\infty} z^n \cdot a_n = z \cdot (\sum_{n=3}^{\infty} z^n \cdot a_n + a_0 - a_0 + \\&
    +a_1 z - a_1 z + a_2 z^2 - a_2 z^2) = z(G(z)-z-z^2) \\&
    \sum_{n=4}^{\infty} z^n \cdot a_{n-2} = z^2 \cdot \sum_{n=4}^{\infty} z^{n-2} \cdot a_{n-2} = z^2 \cdot \sum_{n=2}^{\infty} z^n \cdot a_n = z^2 \cdot (\sum_{n=2}^{\infty} z^n \cdot a_n + a_0 - a_0 + \\&
    +a_1 z - a_1 z) = z^2(G(z)-z) \\&
    \sum_{n=4}^{\infty} z^n \cdot a_{n-3} = z^3 \cdot \sum_{n=4}^{\infty} z^{n-3} \cdot a_{n-3} = z^3 \cdot \sum_{n=1}^{\infty} z^n \cdot a_n = z^3 \cdot (\sum_{n=1}^{\infty} z^n \cdot a_n + a_0 - a_0) = z^3 G(z) \\&
    \sum_{n=4}^{\infty} z^n \cdot a_{n-4} = z^4 \cdot \sum_{n=4}^{\infty} z^{n-4} \cdot a_{n-4} = z^4 \cdot \sum_{n=0}^{\infty} z^n \cdot a_n = z^4 G(z) \\&
    G(z)= z + z^2 + z^3 + 6z(G(z)-z-z^2)-12z^2(G(z)-z)+18z^3G(z)-27z^4G(z) \\&
    G(z)(1-6z+12z^2-18z^3+27z^4)=z+z^2+z^3-6z^2-6z^3+12z^3 \\&
    G(z)=\frac{z-5z^2+7z^3}{1-6z+12z^2-18z^3+27z^4} = \frac{1}{72(1-3z)}+\frac{1}{36(1-3z)^2}+\frac{19z-1}{24(3z^2+1)} \\&
    \frac{1}{72(1-3z)} = \frac{1}{72} \cdot \sum_{n=0}^{\infty} (3z)^n  \\&
    \frac{1}{36(1-3z)^2} = \frac{1}{36} \cdot \sum_{n=0}^{\infty} (3z)^n \cdot (1+n) \\&
    \frac{19z-1}{24(3z^2+1)} = \frac{-1}{16} \cdot \sum_{n=0}^{\infty} (z)^n \cdot 3^{\frac{n-3}{2}}\cdot (-19i(-i)^n+19n^{i+1}+(-i)^n \sqrt(3) + i \sqrt[n](3)) \\&
\end{align}
\begin{align}
    & a_n=\frac{1}{72} \cdot 3^n + \frac{1}{36} \cdot (3)^n \cdot (1+n) - \frac{1}{16} \cdot 3^{\frac{n-3}{2}}\cdot (-19i(-i)^n+19n^{i+1}+(-i)^n \sqrt(3) + i \sqrt[n](3))
\end{align}   
\end{flushleft}

\begin{flushleft}
Ответ: $a_n=\frac{1}{72} \cdot 3^n + \frac{1}{36} \cdot (3)^n \cdot (1+n) - \frac{1}{16} \cdot 3^{\frac{n-3}{2}}\cdot (-19i(-i)^n+19n^{i+1}+(-i)^n \sqrt(3) + i \sqrt[n](3)$.
\end{flushleft}

\section{№18 Найти асимптотическое значение произведения исопльзуя формулу суммирования Эйлера}

\begin{flushleft}
\begin{equation}
\prod_{k=1}^{n} k^k
\end{equation}
Перейдем к решению. Чтобы применить формулу суммирования Эйлера нам нужно перейти от произведения к суммированию. Прологарифмируем:
\begin{align}
\prod_{k=1}^{n} k^k=e^{ln(\prod_{k=1}^{n} k^k)} = e^{\sum_{k=1}^{n} k\cdot ln(k)}
\end{align}
Рассмотрим $\sum_{k=1}^{n} k\cdot ln(k)$. По формуле суммирования Эйлера-Маклорена для вычисления асимптотического выражения для суммы используется следующее приближение:
\begin{align}
& \sum_{n=a}^b f(n) \equiv \int_a^b f(x)dx + \frac{f(a)+f(b)}{2}+\sum_{k=1}^{+\infty} \frac{B_{2k}}{(2k)!} \cdot (f(b)^{2k-1}-f(a)^{2k-1})
\end{align}
Тогда рассматриваемая сумма примет вид:
\begin{align}
& \sum_{k=1}^{n} k\cdot \ln(k) \equiv \int_1^n x\cdot \ln(x) + \frac{n \cdot ln(n)}{2} + \sum_{m=1}^{+\infty} \frac{B_{2m}\cdot (n \cdot \ln(n))^{2m-1}}{(2m)!}
\end{align}
Найдем определенный интеграл:
\begin{align}
& \int_1^n x\cdot ln(x) = \frac{1}{4} \cdot n^2 \cdot (2 ln(n)-1) + \frac{1}{4}
% \\& \sum_{k=1}^{n} k\cdot ln(k) \equiv \frac{n\cdot ln(n)\cdot (n+2)-n^2+1}{4} + \sum_{m=1}^{+\infty} \frac{B_{2m}\cdot (n \cdot ln(n))^{2m-1}}{(2m)!}
\end{align}
Теперь поработаем с бесконечным рядом $\sum_{m=1}^\infty \frac{f^{(2m-1)}(n) \cdot B_{2m}}{(2m)!}$, где $f^{(2m-1)}(n)$ - $(2m-1)$-я производная функции $n \cdot ln(n)$. Рассмотрим данную сумму ряда при $n \rightarrow \infty$.
\begin{align}
& \frac{f(n)^{(1)}}{12} - \frac{f(n)^{(3)}}{720} + \frac{f(n)^{(5)}}{30240} - \frac{f(n)^{(7)}}{1209600} + \frac{f(n)^{(9)}}{47900160} - \cdots
\end{align}
Подставим производные функции $n \cdot ln(n)$:
\begin{align}
& \frac{ln(n)+1}{12} + \frac{1}{n^2 \cdot 720} - \frac{1}{n^4 \cdot 30240} + \frac{1}{n^6 \cdot 1209600} - \frac{1}{n^8 \cdot 47900160} + \cdots
\end{align}
Можно заметить, что полученное значение будет в большинстве своем зависеть только от $\frac{ln(n)+1}{12}$ поскольку сумма остальных членов ряда будет стремиться к нулю. Таким образом: 
\begin{align}
& \sum_{m=1}^\infty \frac{f^{(2m-1)}(n) \cdot B_{2m}}{(2m)!} \equiv \frac{ln(n)+1}{12}
\end{align}
Тогда итоговое преобразование:
\begin{align}   
& \prod_{k=1}^{n} k^k \equiv e^{\frac{n^2 \cdot ln(n)}{2}-\frac{n^2}{4}+\frac{1}{4} + \frac{n \cdot ln(n)}{2}+\frac{ln(n)+1}{12}}
\end{align}
\end{flushleft}

\section{№19 Найти асимптотическое значение произведения исопльзуя формулу суммирования Эйлера}

\begin{flushleft}
    \begin{equation}
    \prod_{k=1}^{n} k^\frac{1}{k}
    \end{equation}
    Перейдем к решению. Чтобы применить формулу суммирования Эйлера нам нужно перейти от произведения к суммированию. Прологарифмируем:
    \begin{align}
    \prod_{k=1}^{n} k^\frac{1}{k}=e^{ln(\prod_{k=1}^{n} k^\frac{1}{k})} = e^{\sum_{k=1}^{n} \frac{ln(k)}{k}}
    \end{align}
    Рассмотрим $\sum_{k=1}^{n} \frac{ln(k)}{k}$. По формуле суммирования Эйлера-Маклорена для вычисления асимптотического выражения для суммы используется следующее приближение:
    \begin{align}
    & \sum_{n=a}^b f(n) \equiv \int_a^b f(x)dx + \frac{f(a)+f(b)}{2}+\sum_{k=1}^{+\infty} \frac{B_{2k}}{(2k)!} \cdot (f(b)^{2k-1}-f(a)^{2k-1})
    \end{align}
    Тогда рассматриваемая сумма примет вид:
    \begin{align}
    & \sum_{k=1}^{n} \frac{ln(k)}{k} \equiv \int_1^n \frac{ln(x)}{x} + \frac{ln(n)}{2n} + \sum_{m=1}^{+\infty} \frac{B_{2m}\cdot (\frac{ln(n)}{n})^{2m-1}}{(2m)!}
    \end{align}
    Найдем определенный интеграл:
    \begin{align}
    & \int_1^n \frac{ln(x)}{x} = \frac{ln^2(n)}{2}
    \end{align}
    Теперь поработаем с бесконечным рядом $\sum_{m=1}^\infty \frac{f^{(2m-1)}(n) \cdot B_{2m}}{(2m)!}$, где $f^{(2m-1)}(n)$ - $(2m-1)$-я производная функции $\frac{ln(n)}{n}$. Рассмотрим данную сумму ряда при $n \rightarrow \infty$.
    \begin{align}
    & \frac{f(n)^{(1)}}{12} - \frac{f(n)^{(3)}}{720} + \frac{f(n)^{(5)}}{30240} - \frac{f(n)^{(7)}}{1209600} + \frac{f(n)^{(9)}}{47900160} - \cdots
    \end{align}
    Подставим производные функции $\frac{ln(n)}{n}$:
    \begin{align}
    & \frac{1-ln(n)}{n^2 \cdot 12} - \frac{11-6 \cdot ln(n)}{n^4 \cdot 720} + \frac{274-120 \cdot ln(n)}{n^6 \cdot 30240} - \frac{13068-5040 \cdot ln(n)}{n^8 \cdot 1209600} + \\&
    + \frac{1026576-362880 \cdot ln(n)}{n^{10} \cdot 47900160} - \cdots
    \end{align}
    Можно заметить, что полученное значение будет в большинстве своем зависеть только от $\frac{1-ln(n)}{12n^2}$ поскольку сумма остальных членов ряда будет стремиться к нулю. Таким образом: 
    \begin{align}
    & \sum_{m=1}^\infty \frac{f^{(2m-1)}(n) \cdot B_{2m}}{(2m)!} \equiv \frac{1-ln(n)}{12n^2}
    \end{align}
    Тогда итоговое преобразование:
    \begin{align}   
    & \prod_{k=1}^{n} k^k \equiv e^{\frac{n^2 \cdot ln(n)}{2}-\frac{n^2}{4}+\frac{1}{4} + \frac{n \cdot ln(n)}{2}+\frac{1-ln(n)}{12n^2}}
    \end{align}
\end{flushleft}

\end{document}